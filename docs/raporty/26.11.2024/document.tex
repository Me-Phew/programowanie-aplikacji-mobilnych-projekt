
%╔════════════════════════════╗
%║		Szablon wykonał		  ║
%║	mgr inż. Dawid Kotlarski  ║
%║		  06.10.2024		  ║
%╚════════════════════════════╝

\documentclass[12pt,a4paper]{mwart}
\usepackage[utf8]{inputenc}
\usepackage{polski}
\usepackage[T1]{fontenc}
\usepackage{amsmath}
\usepackage{amsfonts}
\usepackage{amssymb}
\usepackage{graphicx}
\usepackage{array}
\usepackage{multirow}
\usepackage{geometry}
\usepackage{tabularray}

\geometry{legalpaper, margin=1.5cm}

\renewcommand{\arraystretch}{1.2}

\begin{document}

\begin{center}
	\Huge RAPORT
\end{center}

\begin{table}[h!]
	\centering
	\begin{tblr}{
		width = \linewidth,
		colspec = {Q[156]Q[156]Q[156]Q[156]Q[156]Q[156]},
		row{1} = {c},
		column{4} = {c},
		column{6} = {c},
		cell{1}{1} = {c=6}{0.936\linewidth},
		cell{2}{2} = {c=5}{0.803\linewidth},
		cell{3}{2} = {c=5}{0.803\linewidth},
		cell{4}{2} = {c},
		cell{5}{2} = {c=5}{0.803\linewidth},
		hline{1,6} = {1}{-}{leftpos = 1, rightpos = 1},
		hline{1,6} = {2}{-}{leftpos = 1, rightpos = 1},
		hline{2,2} = {1}{-}{leftpos = 1, rightpos = 1},
		hline{2,2} = {2}{-}{leftpos = 1, rightpos = 1},
		vline{1,1} = {1}{-}{abovepos = 1, belowpos = 1},
		vline{1,1} = {2}{-}{abovepos = 1, belowpos = 1},
		vline{7,1} = {1}{-}{abovepos = 1, belowpos = 1},
		vline{7,1} = {2}{-}{abovepos = 1, belowpos = 1},
				hlines,
				vlines,
			}
		{AKADEMIA NAUK STOSOWANYCH W NOWYM SĄCZU                                                                                 \\Wydział Nauk Inżynieryjnych, Katedra informatyki} &  &  &  &  &  \\
		Przedmiot: & Programowanie urządzeń mobilnych – projekt, mgr inż. Dawid Kotlarski &             &   &       &            \\
		Temat:     & Wirtualny Dziekanat                                                  &             &   &       &            \\
		Grupa:     & IS-2(s)P1                                                            & Nr raportu: & 6 & Data: & 26.11.2024 \\
		Osoby:     & Marcin Dudek, Mateusz Basiaga                                        &             &   &       &
	\end{tblr}
\end{table}

\section{Wykonane zadania}
\begin{itemize}
	\item Migracja backendu do nowej wersji Payload CMS (3.0),
	\item Połączenie nowego backendu z aplikacją,
	\item Usunięcie ekranu rejestracji (tylko administratorzy mogą tworzyć konta studentów / pracowników),
\end{itemize}

\section{Niewykonane zadania}
\begin{itemize}
	\item Dokończenie ekranu wysyłania wiadomości,
	\item Dodanie powiadomień o zmianie harmonogramu zajęć.
\end{itemize}

\section{Napotkane problemy}
\textit{Brak}


\section{Zadania na kolejny tydzień}
\begin{itemize}
	\item Dokończenie ekranu wysyłania wiadomości,
	\item Dodanie powiadomień o zmianie harmonogramu zajęć,
	\item Zapisywanie danych użytkownika lokalnie,
	\item Dostęp do aplikacji bez połączenia z Internetem.
\end{itemize}

\section{Stan dokumentacji projektowej}
\begin{itemize}
	\item Rozpoczęto dokumentowanie implementacji (rozdział 4).
\end{itemize}

\end{document}
