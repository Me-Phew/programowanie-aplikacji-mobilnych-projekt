\newpage
\section{Implementacja}		%4
%Wkleić szkielet kodu, wraz z komentarzami. Opisać zmienne, struktury do czego służą. Opisać procedury, metody co wykonują. Opisać nowe zdefiniowane klasy. Opisać dziedziczenie. Opisać nowo utworzone pliki za co odpowiadają.

\subsection{Struktura projektu}

Projekt został podzielony na następujące główne katalogi:

\begin{itemize}
  \item \textbf{lib/} - główny katalog z kodem źródłowym
        \begin{itemize}
          \item \textbf{models/} - modele danych
          \item \textbf{screens/} - widoki aplikacji
          \item \textbf{utils/} - narzędzia pomocnicze
          \item \textbf{providers/} - zarządzanie stanem
          \item \textbf{services/} - warstwa komunikacji z API
          \item \textbf{widgets/} - komponenty wielokrotnego użytku
        \end{itemize}
\end{itemize}

\subsection{Modele danych}

\subsubsection{Model użytkownika \textbf{Listing. \ref{lst:user-model} (s. \pageref{lst:user-model})}}
\begin{lstlisting}[language=C++, caption=Model użytkownika, label={lst:user-model}]
class AppUser {
  final String uid;
  final String email;
  
  AppUser({
    required this.uid,
    required this.email,
  });
}
\end{lstlisting}
Model `AppUser` reprezentuje użytkownika aplikacji. Zawiera unikalny identyfikator `uid` oraz adres email `email`.
\newpage
\subsubsection{Model wydarzenia \textbf{Listing. \ref{lst:event-model} (s. \pageref{lst:event-model})}}
\begin{lstlisting}[language=C++, caption=Model wydarzenia, label={lst:event-model}]
class Event {
  final String title;
  final TimeOfDay startTime;
  final TimeOfDay endTime;
  final String room;
  final String type;
  final String lecturer;
  final DateTime date;

  Event({
    required this.title,
    required this.startTime,
    required this.endTime,
    required this.room,
    required this.lecturer,
    required this.type,
    required this.date,
  });
}
\end{lstlisting}
Model `Event` reprezentuje wydarzenie w aplikacji. Zawiera informacje takie jak tytuł `title`, czas rozpoczęcia `startTime`, czas zakończenia `endTime`, sala `room`, typ zajęć `type`, wykładowca `lecturer` oraz data `date`.

\subsection{Zarządzanie stanem}

\subsubsection{Dostawca motywu \textbf{Listing. \ref{lst:theme-provider} (s. \pageref{lst:theme-provider})}}
\begin{lstlisting}[language=C++, caption=Provider motywu, label={lst:theme-provider}]
final darkModeProvider = StateNotifierProvider<DarkModeNotifier, bool>((ref) {
  return DarkModeNotifier();
});

class DarkModeNotifier extends StateNotifier<bool> {
  DarkModeNotifier() : super(false);

  void toggleDarkMode(bool value) {
    state = value;
  }
}
\end{lstlisting}
Provider `darkModeProvider` zarządza stanem trybu ciemnego w aplikacji. Klasa `DarkModeNotifier` dziedziczy po `StateNotifier<bool>` i umożliwia przełączanie trybu ciemnego za pomocą metody `toggleDarkMode`.

\subsection{Usługi API}

\subsubsection{Serwis powiadomień \textbf{Listing. \ref{lst:notification-service} (s. \pageref{lst:notification-service})}}
\begin{lstlisting}[language=C++, caption=Serwis powiadomień, label={lst:notification-service}]
class FirebaseApi {
  final _firebaseMessaging = FirebaseMessaging.instance;

  Future<void> initNotifications() async {
    await _firebaseMessaging.requestPermission();
    final token = await _firebaseMessaging.getToken();
    print('Token: $token');
  }

  void handleMessage(RemoteMessage? message) {
    if (message == null) return;
  }
}
\end{lstlisting}
Klasa `FirebaseApi` odpowiada za obsługę powiadomień push. Metoda `initNotifications` inicjalizuje powiadomienia, a `handleMessage` obsługuje otrzymane wiadomości.

\subsection{Komponenty wielokrotnego użytku}

\subsubsection{Przycisk stylizowany \textbf{Listing. \ref{lst:styled-button} (s. \pageref{lst:styled-button})}}
\begin{lstlisting}[language=C++, caption=Komponent przycisku, label={lst:styled-button}]
class StyledButton extends StatelessWidget {
  final VoidCallback onPressed;
  final Widget child;

  const StyledButton({
    Key? key,
    required this.onPressed,
    required this.child,
  }) : super(key: key);

  @override
  Widget build(BuildContext context) {
    return ElevatedButton(
      onPressed: onPressed,
      style: ElevatedButton.styleFrom(
        shape: RoundedRectangleBorder(
          borderRadius: BorderRadius.circular(12),
        ),
      ),
      child: child,
    );
  }
}
\end{lstlisting}
Komponent `StyledButton` to stylizowany przycisk, który można wielokrotnie używać w aplikacji. Przyjmuje funkcję `onPressed` oraz widget `child` jako argumenty.

\subsection{Logika biznesowa}

\subsubsection{Filtrowanie wydarzeń \textbf{Listing. \ref{lst:event-filtering} (s. \pageref{lst:event-filtering})}}
\begin{lstlisting}[language=C++, caption=Filtrowanie wydarzeń, label={lst:event-filtering}]
List<Event> filterEvents(List<Event> events, DateTime selectedDay) {
  return events
    .where((event) => isSameDay(event.date, selectedDay))
    .toList()
    ..sort((a, b) {
      final aStart = DateTime(
        selectedDay.year, 
        selectedDay.month, 
        selectedDay.day,
        a.startTime.hour,
        a.startTime.minute
      );
      final bStart = DateTime(
        selectedDay.year,
        selectedDay.month,
        selectedDay.day,
        b.startTime.hour,
        b.startTime.minute
      );
      return aStart.compareTo(bStart);
    });
}
\end{lstlisting}
Funkcja `filterEvents` filtruje i sortuje wydarzenia na podstawie wybranego dnia `selectedDay`.

\subsection{Wygodne SDK do komunikacji z backendem}

Poniższy kod implementuje klasę \texttt{WirtualnySdk}, która pełni rolę wygodnego narzędzia do komunikacji z backendem. Dzięki wykorzystaniu wzorców projektowych takich jak \textbf{Singleton} oraz \textbf{Factory}, zapewniono efektywną inicjalizację i dostęp do komponentów SDK.

\subsubsection{Wzorzec Singleton}
Wzorzec \textbf{Singleton} umożliwia istnienie tylko jednej instancji klasy \texttt{WirtualnySdk} w całej aplikacji. Dzięki temu wszystkie moduły aplikacji mają dostęp do tej samej konfiguracji SDK i mogą współdzielić zasoby. W kodzie poniżej mechanizm Singleton realizowany jest poprzez:

\begin{itemize}
  \item Prywatne pole statyczne \texttt{\_instance}, które przechowuje instancję klasy.
  \item Prywatny konstruktor \texttt{\_internal}, który zapobiega tworzeniu instancji z zewnątrz.
  \item Publiczną metodę \texttt{initialize()}, która inicjalizuje SDK i ustawia konfigurację.
  \item Getter \texttt{instance}, który zwraca istniejącą instancję SDK lub zgłasza wyjątek, jeśli SDK nie zostało zainicjalizowane.
\end{itemize}

Kod weryfikuje, czy SDK zostało już zainicjalizowane. Próba wielokrotnej inicjalizacji prowadzi do zgłoszenia wyjątku \textbf{Listing. \ref{lst:event-sdk} (s. \pageref{lst:event-sdk})}:

\begin{lstlisting}[language=C++, caption=Zgłoszenie wyjątku,  label={lst:event-sdk}]
if (_instance != null) {
    throw Exception('WirtualnySdk has already been initialized.');
}
\end{lstlisting}

\subsubsection{Wzorzec Factory}
Wzorzec \textbf{Factory} umożliwia tworzenie instancji klasy poprzez wywołanie metody \texttt{factory}. W kodzie zastosowano fabrykę do implementacji uproszczonego dostępu do obiektu SDK \textbf{Listing. \ref{lst:event-factory} (s. \pageref{lst:event-factory})}:

\begin{lstlisting}[language=C++, caption=Wzorzec Factory,  label={lst:event-factory}]
factory WirtualnySdk() {
    return instance;
}
\end{lstlisting}

Fabryka zwraca już istniejącą instancję klasy \texttt{WirtualnySdk}, co integruje się z mechanizmem Singleton. Umożliwia to bezpieczny i wygodny dostęp do SDK bez potrzeby ręcznego zarządzania instancją.

\subsubsection{Modularna budowa SDK}
Klasa \texttt{WirtualnySdk} zapewnia dostęp do modułów \texttt{auth} oraz \texttt{notifications}, odpowiedzialnych za uwierzytelnianie i obsługę powiadomień. Każdy z modułów jest instancjonowany jako prywatne pole i dostępny za pośrednictwem getterów \textbf{Listing. \ref{lst:event-auth} (s. \pageref{lst:event-auth})}:

\begin{lstlisting}[language=C++, caption=WirtualnyAuth,  label={lst:event-auth}]
final WirtualnyAuth _auth = WirtualnyAuth();
final WirtualnyNotifications _wirtualnyNotifications =
    WirtualnyNotifications();

WirtualnyAuth get auth => _auth;
WirtualnyNotifications get notifications => _wirtualnyNotifications;
\end{lstlisting}

Dzięki takiemu podejściu aplikacja ma zapewnioną modularność i łatwość rozbudowy SDK o kolejne komponenty.

\subsubsection{Podsumowanie}
Implementacja klasy \texttt{WirtualnySdk} pokazuje skuteczne wykorzystanie wzorców Singleton i Factory, dzięki czemu SDK jest łatwe w użyciu, bezpieczne i zgodne z zasadami dobrej architektury. Klasa zapewnia wygodny dostęp do kluczowych funkcji, takich jak uwierzytelnianie i powiadomienia, co czyni ją centralnym punktem integracji z backendem.

\subsection{Wysyłanie powiadomień o ogłoszeniach - implementacja backendowa}

Poniższy kod w języku TypeScript przedstawia proces wysyłania powiadomień typu FCM (Firebase Cloud Messaging) do wielu urządzeń jednocześnie w formie powiadomień o ogłoszeniach. Obsługiwane są zarówno urządzenia Android, jak i iOS. Kod zawiera również mechanizm logowania oraz obsługi błędów. \textbf{Listing. \ref{lst:fcm_notification} (s. \pageref{lst:fcm_notification})}

\begin{lstlisting}[language=c++, caption={Implementacja wysyłania powiadomień FCM}, label={lst:fcm_notification}]
try {
  getMessaging().sendEachForMulticast({
    tokens: fcmTokens,
    notification: {
      title: doc.subject,
      body: stripHtml(doc.content_html).result,
    },
    data: {
      type: 'announcement',
      id: doc.id.toString(),
      click_action: 'FLUTTER_NOTIFICATION_CLICK',
    },
    android: {
      priority: doc.priority === 'high' ? 'high' : 'normal',
      notification: {
        clickAction: 'FLUTTER_NOTIFICATION_CLICK',
      },
    },
    apns: {
      headers: {
        'apns-priority': doc.priority === 'high' ? '10' : '5',
      },
    },
    fcmOptions: {
      analyticsLabel: 'announcement',
    },
  });

  if (enableLogs) {
    payload.logger.info(
      `Successfully sent FCM notification for '${collectionSlug}' document with ID: '${doc.id}''.`,
    );
  }
} catch (error: unknown) {
  const msg = error instanceof Error ? error.message : error;
  throw new APIError(
    `Failed to send FCM notification for '${collectionSlug}' document with ID: '${doc.id}': ${msg}'.`,
  );
}
\end{lstlisting}

\textbf{Opis działania:}
\begin{itemize}
  \item \textbf{Powiadomienia FCM:} Funkcja \texttt{sendEachForMulticast} umożliwia wysyłanie powiadomień do grupy urządzeń zdefiniowanych w tablicy \texttt{fcmTokens}.
  \item \textbf{Struktura powiadomienia:} Powiadomienia zawierają tytuł (\texttt{title}), treść (\texttt{body}) oraz dane dodatkowe (\texttt{data}), takie jak typ powiadomienia (\texttt{announcement}) czy identyfikator dokumentu (\texttt{id}).
  \item \textbf{Obsługa platform:}
        \begin{itemize}
          \item \textbf{Android:} Ustawiane są priorytety (\texttt{high} lub \texttt{normal}) oraz akcje kliknięcia.
          \item \textbf{iOS:} Nagłówki \texttt{apns-priority} definiują priorytet wiadomości.
        \end{itemize}
  \item \textbf{Logowanie:} W przypadku sukcesu wysyłania, logowana jest informacja o wysłaniu powiadomienia.
  \item \textbf{Obsługa błędów:} W przypadku błędu, rzucany jest wyjątek \texttt{APIError} z odpowiednim komunikatem.
\end{itemize}

\subsection{Integracja z backendem}

Aplikacja komunikuje się z backendem Payload CMS poprzez REST API. Główne endpointy:

\begin{itemize}
  \item \texttt{/api/auth/login} - logowanie użytkownika
  \item \texttt{/api/events} - zarządzanie planem zajęć
  \item \texttt{/api/subjects} - zarządzanie przedmiotami
  \item \texttt{/api/notifications} - zarządzanie powiadomieniami
\end{itemize}
Uwierzytelnianie odbywa się poprzez tokeny JWT przechowywane w \texttt{SharedPreferences}.

\subsection{Deployment}

Projekt PayloadCMS\cite{github-backend} został wdrożony na serwerze Ubuntu\cite{backend} z wykorzystaniem Nginx jako reverse proxy, certyfikatu SSL od Certbota, zarządzania DNS przez Cloudflare oraz narzędzi takich jak PM2 i pnpm. PayloadCMS w wersji 3 działa na frameworku Next.js, co zapewnia wydajność i elastyczność w budowie aplikacji.

\subsubsection{Uruchamianie aplikacji za pomocą pnpm i PM2}

Do zarządzania zależnościami projektu wykorzystano pnpm, co pozwala na szybsze instalacje oraz oszczędność miejsca na dysku. W katalogu projektu wszystkie niezbędne zależności zostały zainstalowane poleceniem \textbf{Listing. \ref{lst:event-pnpm} (s. \pageref{lst:event-pnpm})}:
\begin{lstlisting}[language=C++, caption=Instalacja pnpm,  label={lst:event-pnpm}]
pnpm install
\end{lstlisting}

Po zakończeniu konfiguracji i upewnieniu się, że aplikacja działa lokalnie w trybie produkcyjnym (\texttt{pnpm build} oraz \texttt{pnpm start}), PayloadCMS został uruchomiony i monitorowany za pomocą PM2. PM2 jest używany do zarządzania procesami aplikacji w tle, co zapewnia niezawodność oraz możliwość automatycznego restartu w razie awarii. Aplikację uruchomiono w trybie produkcyjnym następująco \textbf{Listing. \ref{lst:event-pnpm2} (s. \pageref{lst:event-pnpm2})}:
\begin{lstlisting}[language=C++, caption=Uruchamianie pnpm,  label={lst:event-pnpm2}]
pm2 start npm --name "payload-cms" -- start
\end{lstlisting}

PM2 zapewnia również możliwość monitorowania procesów oraz logów \textbf{Listing. \ref{lst:event-pnpm3} (s. \pageref{lst:event-pnpm3})}:
\begin{lstlisting}[language=C++, caption=Sprawdzenie status oraz logów przy pomocy pnpm,  label={lst:event-pnpm3}]
pm2 status
pm2 logs payload-cms
\end{lstlisting}

\subsubsection{Konfiguracja Nginx jako Reverse Proxy}

Nginx został skonfigurowany jako reverse proxy, aby przekierowywać ruch HTTP/HTTPS na serwer PayloadCMS działający na porcie aplikacji. Oto przykład konfiguracji wirtualnego hosta \textbf{Listing. \ref{lst:event-pnpm4} (s. \pageref{lst:event-pnpm4})}:
\begin{lstlisting}[language=C++, caption=Konfiguracja nginx,  label={lst:event-pnpm4}]
server {
    listen 80;
    server_name example.com www.example.com;

    location / {
        proxy_pass http://localhost:3000;
        proxy_http_version 1.1;
        proxy_set_header Upgrade $http_upgrade;
        proxy_set_header Connection 'upgrade';
        proxy_set_header Host $host;
        proxy_cache_bypass $http_upgrade;
    }
}
\end{lstlisting}

Po dodaniu powyższej konfiguracji Nginx został przeładowany \textbf{Listing. \ref{lst:event-pnpm5} (s. \pageref{lst:event-pnpm5})}:
\begin{lstlisting}[language=C++, caption=przeładowanie konfiguracji,  label={lst:event-pnpm5}]
sudo systemctl reload nginx
\end{lstlisting}

\subsubsection{SSL i Cloudflare}

Domenę skonfigurowano z użyciem Cloudflare oraz certyfikatu SSL wygenerowanego przez Certbota. Certbot zaktualizował konfigurację Nginx w celu obsługi HTTPS \textbf{Listing. \ref{lst:event-pnpm6} (s. \pageref{lst:event-pnpm6})}:
\begin{lstlisting}[language=C++, caption=Ustawienie cerbota,  label={lst:event-pnpm6}]
sudo certbot --nginx -d example.com -d www.example.com
\end{lstlisting}

Cloudflare zostało skonfigurowane do zarządzania DNS (rekordy \texttt{A} i \texttt{CNAME}) oraz aktywowano tryb \texttt{Full SSL/TLS encryption mode}.

\subsubsection{Podsumowanie}

Dzięki wykorzystaniu pnpm do zarządzania zależnościami oraz PM2 do nadzorowania procesu aplikacji, PayloadCMS w wersji 3 (opartej na Next.js) działa stabilnie i wydajnie. Dodatkowo Nginx, Certbot oraz Cloudflare zapewniają bezpieczeństwo, wysoką dostępność i optymalizację dla użytkowników końcowych.

\subsection{Przygotowanie pakietu .apk aplikacji mobilnej}

Proces przygotowania pakietu \texttt{.apk} dla aplikacji mobilnej stworzonej w Flutterze obejmuje kilka kroków, od konfiguracji projektu po generowanie gotowego pliku instalacyjnego. Poniżej przedstawiono szczegóły całego procesu.

\subsubsection{Konfiguracja projektu Flutter}

Przed rozpoczęciem procesu budowy upewniono się, że projekt Flutter jest odpowiednio skonfigurowany do pracy w środowisku produkcyjnym:
\begin{enumerate}
  \item Plik \texttt{pubspec.yaml} został zaktualizowany, a wszystkie zależności zostały zainstalowane \textbf{Listing. \ref{lst:event-pnpm7} (s. \pageref{lst:event-pnpm7})}:
        \begin{lstlisting}[language=C++, caption=Pobranie pakietów w flutterze,  label={lst:event-pnpm7}]
    flutter pub get
    \end{lstlisting}
  \item Zdefiniowano odpowiednią konfigurację w pliku \texttt{AndroidManifest.xml}, w tym \texttt{applicationId}, pozwolenia (\texttt{permissions}) oraz ustawienia dotyczące wersji aplikacji (\texttt{versionCode} i \texttt{versionName}).
  \item Plik \texttt{build.gradle} w module \texttt{app} zawierał właściwą konfigurację dla środowiska produkcyjnego, w tym \texttt{minSdkVersion} i \texttt{targetSdkVersion}.
\end{enumerate}

\subsubsection{Budowanie pakietu .apk}

Pakiet \texttt{.apk} został wygenerowany za pomocą polecenia Fluttera do budowy w trybie produkcyjnym \textbf{Listing. \ref{lst:event-pnpm8} (s. \pageref{lst:event-pnpm8})}:
\begin{lstlisting}[language=C++, caption=Budowa oficjalnego pakietu .apk,  label={lst:event-pnpm8}]
flutter build apk --release
\end{lstlisting}

Wygenerowany plik \texttt{app-release.apk} został zapisany w katalogu \texttt{build/app/outputs/flutter-apk/}.

\subsubsection{Podpisywanie aplikacji}

Aby aplikacja mogła być zainstalowana na urządzeniach użytkowników, wymagane było jej podpisanie:
\begin{enumerate}
  \item Wygenerowano klucz kryptograficzny (\texttt{.keystore}) za pomocą narzędzia \texttt{keytool} \textbf{Listing. \ref{lst:event-pnpm9} (s. \pageref{lst:event-pnpm9})}:
        \begin{lstlisting}[language=C++, caption=Generowanie kluczu krypotraficznego,  label={lst:event-pnpm9}]
    keytool -genkey -v -keystore my-release-key.jks \
        -keyalg RSA -keysize 2048 -validity 10000 \
        -alias my-key-alias
    \end{lstlisting}
  \item Plik \texttt{my-release-key.jks} został umieszczony w katalogu \texttt{android/app/}.
  \item W pliku \texttt{android/key.properties} zdefiniowano dane dostępowe do klucza \textbf{Listing. \ref{lst:event-pnpm10} (s. \pageref{lst:event-pnpm10})}:
        \begin{lstlisting}[language=C++, caption=Dane dostępowe do klucza,  label={lst:event-pnpm10}]
    storePassword=your-store-password
    keyPassword=your-key-password
    keyAlias=my-key-alias
    storeFile=path/to/my-release-key.jks
    \end{lstlisting}
  \item Plik \texttt{build.gradle} został zaktualizowany, aby uwzględnić konfigurację podpisu \textbf{Listing. \ref{lst:event-pnpm11} (s. \pageref{lst:event-pnpm11})}:
        \begin{lstlisting}[language=C++, caption=Uwzględnienie podpisu w pliku build.gradle,  label={lst:event-pnpm11}]
    android {
        signingConfigs {
            release {
                keyAlias 'my-key-alias'
                keyPassword 'your-key-password'
                storeFile file('path/to/my-release-key.jks')
                storePassword 'your-store-password'
            }
        }
        buildTypes {
            release {
                signingConfig signingConfigs.release
            }
        }
    }
    \end{lstlisting}
\end{enumerate}

\subsubsection{Testowanie pakietu}

Gotowy pakiet \texttt{.apk} został przetestowany na urządzeniach fizycznych i emulatorach, aby upewnić się, że działa poprawnie. Weryfikację przeprowadzono za pomocą polecenia \textbf{Listing. \ref{lst:event-pnpm12} (s. \pageref{lst:event-pnpm12})}:
\begin{lstlisting}[language=C++, caption=Testowanie pakietu,  label={lst:event-pnpm12}]
flutter install
\end{lstlisting}

\subsubsection{Podsumowanie}

Pakiet \texttt{.apk} aplikacji mobilnej został przygotowany i podpisany, co pozwala na jego dystrybucję do użytkowników poprzez ręczną instalację lub publikację w sklepie Google Play. Wykorzystanie Fluttera oraz właściwa konfiguracja środowiska produkcyjnego gwarantują wysoką wydajność oraz zgodność z różnymi urządzeniami Android.
