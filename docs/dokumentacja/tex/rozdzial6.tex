\newpage
\section{Podręcznik użytkownika}  %6
%Opis jak używać programu. Mogą być z zrzut ekranu razem z opisem. 
\subsection{Logowanie}

Po uruchomieniu aplikacji po raz pierwszy użytkownik zostanie przekierowany na ekran logowania. Należy wprowadzić adres e-mail oraz hasło, a następnie kliknąć przycisk "Zaloguj".

% \begin{figure}[h!]
%   \centering
%   \includegraphics[width=0.5\textwidth]{screenshots/login_screen.png}
%   \caption{Ekran logowania}
%   \label{fig:login_screen}
% \end{figure}

\subsection{Ekran główny}

Po zalogowaniu użytkownik zostanie przekierowany na ekran główny. Na ekranie głównym znajduje się plan zajęć, który pokazuje nadchodzące zajęcia. Kliknięcie na konkretne zajęcia wyświetli szczegóły, takie jak sala, wykładowca i godziny.

% \begin{figure}[h!]
%   \centering
%   \includegraphics[width=0.5\textwidth]{screenshots/home_screen.png}
%   \caption{Ekran główny}
%   \label{fig:home_screen}
% \end{figure}

\subsection{Powiadomienia}

Aby zobaczyć powiadomienia, należy kliknąć na ikonę powiadomień w dolnym pasku nawigacyjnym. Wyświetli się lista powiadomień, takich jak nadchodzące zajęcia, nowe oceny oraz ważne ogłoszenia. Kliknięcie na powiadomienie wyświetli szczegóły.

% \begin{figure}[h!]
%   \centering
%   \includegraphics[width=0.5\textwidth]{screenshots/notifications_screen.png}
%   \caption{Powiadomienia}
%   \label{fig:notifications_screen}
% \end{figure}

\subsection{Wiadomości}

Aby zobaczyć wiadomości, należy kliknąć na ikonę wiadomości w dolnym pasku nawigacyjnym. Wyświetli się lista wiadomości od wykładowców i administracji. Kliknięcie na wiadomość wyświetli jej treść.

% \begin{figure}[h!]
%   \centering
%   \includegraphics[width=0.5\textwidth]{screenshots/messages_screen.png}
%   \caption{Wiadomości}
%   \label{fig:messages_screen}
% \end{figure}

\subsection{Ustawienia}

Aby przejść do ustawień, należy kliknąć na ikonę ustawień w dolnym pasku nawigacyjnym. W ustawieniach użytkownik może:

\begin{itemize}
  \item Włączyć lub wyłączyć tryb ciemny
  \item Zarządzać powiadomieniami
  \item Kliknąć na swój avatar, aby zmienić zdjęcie profilowe, zobaczyć informacje o sobie lub wylogować się
\end{itemize}

% \begin{figure}[h!]
%   \centering
%   \includegraphics[width=0.5\textwidth]{screenshots/settings_screen.png}
%   \caption{Ustawienia}
%   \label{fig:settings_screen}
% \end{figure}

\subsubsection{Tryb ciemny}

Aby włączyć tryb ciemny, należy przełączyć odpowiedni przełącznik w ustawieniach. Aplikacja automatycznie zmieni motyw na ciemny.

% \begin{figure}[h!]
%   \centering
%   \includegraphics[width=0.5\textwidth]{screenshots/dark_mode.png}
%   \caption{Tryb ciemny}
%   \label{fig:dark_mode}
% \end{figure}

\subsubsection{Zarządzanie powiadomieniami}

W ustawieniach można włączyć lub wyłączyć powiadomienia dla różnych typów zdarzeń, takich jak nadchodzące zajęcia czy nowe oceny.

% \begin{figure}[h!]
%   \centering
%   \includegraphics[width=0.5\textwidth]{screenshots/notification_settings.png}
%   \caption{Zarządzanie powiadomieniami}
%   \label{fig:notification_settings}
% \end{figure}

\subsubsection{Profil użytkownika}

Kliknięcie na avatar w ustawieniach przenosi użytkownika do ekranu profilu, gdzie można:

\begin{itemize}
  \item Wylogować się z aplikacji
  \item Zmienić zdjęcie profilowe
  \item Zobaczyć informacje o sobie, takie jak imię, nazwisko, adres e-mail
\end{itemize}

% \begin{figure}[h!]
%   \centering
%   \includegraphics[width=0.5\textwidth]{screenshots/profile_screen.png}
%   \caption{Profil użytkownika}
%   \label{fig:profile_screen}
% \end{figure}

\subsubsection{Wylogowanie}

Aby się wylogować, należy kliknąć na przycisk "Wyloguj się" na ekranie profilu. Użytkownik zostanie wylogowany i przekierowany na ekran logowania.

% \begin{figure}[h!]
%   \centering
%   \includegraphics[width=0.5\textwidth]{screenshots/logout_screen.png}
%   \caption{Wylogowanie}
%   \label{fig:logout_screen}
% \end{figure}
