\newpage
\section{Implementacja}		%4
%Wkleić szkielet kodu, wraz z komentarzami. Opisać zmienne, struktury do czego służą. Opisać procedury, metody co wykonują. Opisać nowe zdefiniowane klasy. Opisać dziedziczenie. Opisać nowo utworzone pliki za co odpowiadają.

\subsection{Struktura projektu}

Projekt został podzielony na następujące główne katalogi:

\begin{itemize}
  \item \textbf{lib/} - główny katalog z kodem źródłowym
        \begin{itemize}
          \item \textbf{models/} - modele danych
          \item \textbf{screens/} - widoki aplikacji
          \item \textbf{utils/} - narzędzia pomocnicze
          \item \textbf{providers/} - zarządzanie stanem
          \item \textbf{services/} - warstwa komunikacji z API
          \item \textbf{widgets/} - komponenty wielokrotnego użytku
        \end{itemize}
\end{itemize}

\subsection{Modele danych}

\subsubsection{Model użytkownika}
\begin{lstlisting}[language=C++, caption=Model użytkownika, label={lst:user-model}]
class AppUser {
  final String uid;
  final String email;
  
  AppUser({
    required this.uid,
    required this.email,
  });
}
\end{lstlisting}
Model `AppUser` reprezentuje użytkownika aplikacji. Zawiera unikalny identyfikator `uid` oraz adres email `email`.
\newpage
\subsubsection{Model wydarzenia}
\begin{lstlisting}[language=C++, caption=Model wydarzenia, label={lst:event-model}]
class Event {
  final String title;
  final TimeOfDay startTime;
  final TimeOfDay endTime;
  final String room;
  final String type;
  final String lecturer;
  final DateTime date;

  Event({
    required this.title,
    required this.startTime,
    required this.endTime,
    required this.room,
    required this.lecturer,
    required this.type,
    required this.date,
  });
}
\end{lstlisting}
Model `Event` reprezentuje wydarzenie w aplikacji. Zawiera informacje takie jak tytuł `title`, czas rozpoczęcia `startTime`, czas zakończenia `endTime`, sala `room`, typ zajęć `type`, wykładowca `lecturer` oraz data `date`.

\subsection{Zarządzanie stanem}

\subsubsection{Dostawca motywu}
\begin{lstlisting}[language=C++, caption=Provider motywu, label={lst:theme-provider}]
final darkModeProvider = StateNotifierProvider<DarkModeNotifier, bool>((ref) {
  return DarkModeNotifier();
});

class DarkModeNotifier extends StateNotifier<bool> {
  DarkModeNotifier() : super(false);

  void toggleDarkMode(bool value) {
    state = value;
  }
}
\end{lstlisting}
Provider `darkModeProvider` zarządza stanem trybu ciemnego w aplikacji. Klasa `DarkModeNotifier` dziedziczy po `StateNotifier<bool>` i umożliwia przełączanie trybu ciemnego za pomocą metody `toggleDarkMode`.

\subsection{Usługi API}

\subsubsection{Serwis powiadomień}
\begin{lstlisting}[language=C++, caption=Serwis powiadomień, label={lst:notification-service}]
class FirebaseApi {
  final _firebaseMessaging = FirebaseMessaging.instance;

  Future<void> initNotifications() async {
    await _firebaseMessaging.requestPermission();
    final token = await _firebaseMessaging.getToken();
    print('Token: $token');
  }

  void handleMessage(RemoteMessage? message) {
    if (message == null) return;
  }
}
\end{lstlisting}
Klasa `FirebaseApi` odpowiada za obsługę powiadomień push. Metoda `initNotifications` inicjalizuje powiadomienia, a `handleMessage` obsługuje otrzymane wiadomości.

\subsection{Komponenty wielokrotnego użytku}

\subsubsection{Przycisk stylizowany}
\begin{lstlisting}[language=C++, caption=Komponent przycisku, label={lst:styled-button}]
class StyledButton extends StatelessWidget {
  final VoidCallback onPressed;
  final Widget child;

  const StyledButton({
    Key? key,
    required this.onPressed,
    required this.child,
  }) : super(key: key);

  @override
  Widget build(BuildContext context) {
    return ElevatedButton(
      onPressed: onPressed,
      style: ElevatedButton.styleFrom(
        shape: RoundedRectangleBorder(
          borderRadius: BorderRadius.circular(12),
        ),
      ),
      child: child,
    );
  }
}
\end{lstlisting}
Komponent `StyledButton` to stylizowany przycisk, który można wielokrotnie używać w aplikacji. Przyjmuje funkcję `onPressed` oraz widget `child` jako argumenty.

\subsection{Logika biznesowa}

\subsubsection{Filtrowanie wydarzeń}
\begin{lstlisting}[language=C++, caption=Filtrowanie wydarzeń, label={lst:event-filtering}]
List<Event> filterEvents(List<Event> events, DateTime selectedDay) {
  return events
    .where((event) => isSameDay(event.date, selectedDay))
    .toList()
    ..sort((a, b) {
      final aStart = DateTime(
        selectedDay.year, 
        selectedDay.month, 
        selectedDay.day,
        a.startTime.hour,
        a.startTime.minute
      );
      final bStart = DateTime(
        selectedDay.year,
        selectedDay.month,
        selectedDay.day,
        b.startTime.hour,
        b.startTime.minute
      );
      return aStart.compareTo(bStart);
    });
}
\end{lstlisting}
Funkcja `filterEvents` filtruje i sortuje wydarzenia na podstawie wybranego dnia `selectedDay`.

\subsection{Integracja z backendem}

Aplikacja komunikuje się z backendem Payload CMS poprzez REST API. Główne endpointy:

\begin{itemize}
  \item \texttt{/api/auth/login} - logowanie użytkownika
  \item \texttt{/api/events} - zarządzanie planem zajęć
  \item \texttt{/api/messages} - system wiadomości
  \item \texttt{/api/notifications} - zarządzanie powiadomieniami
\end{itemize}
Uwierzytelnianie odbywa się poprzez tokeny JWT przechowywane w \texttt{SharedPreferences}.
