\newpage
\section{Ogólne określenie wymagań}		%1
%Ogólne określenie wymagań i zakresu programu (Czyli zleceniodawca określa wymagania programu) 

\subsection{Główne funkcje aplikacji}

\begin{itemize}
	\item \textbf{Logowanie i autoryzacja:}
	      \\Użytkownicy powinni móc logować się za pomocą uczelnianego adresu e-mail oraz hasła. Po piewrszym logowaniu użytkownik dostanie opcje logowania za się pomocą odcisku palaca albo skanu twarzy. Możliwość logowania tyczy się zarówno dla studentów, jak i pracowników (np. wykładowcy, administracja).
	\item \textbf{Podgląd ocen i zaliczeń:}
	      \\Studenci powinni mieć możliwość przeglądania swoich ocen z egzaminów, kolokwiów oraz innych zaliczeń. Możliwość filtrowania wyników na podstawie przedmiotu, semestru czy wykładowcy.
	\item \textbf{Plan zajęć:}
	      \\Podgląd bieżącego planu zajęć z opcją aktualizacji na żywo (jeśli np. zajęcia zostaną odwołane czy przeniesione).
	\item \textbf{Harmonogram egzaminów i sesji:}
	      \\Informacje o nadchodzących egzaminach, sesjach poprawkowych i innych ważnych wydarzeniach związanych z uczelnią. Możliwość zapisywania się na egzaminy, jeżeli to wymagane.
	\item \textbf{Powiadomienia:}
	      \\Push notifications o nowych ocenach, nadchodzących zajęciach, zmianach w harmonogramie lub ważnych ogłoszeniach.
	\item \textbf{Informacje ogólne:}
	      \\Tablica ogłoszeń z najważniejszymi informacjami od uczelni, np. nowe zarządzenia rektora, wydarzenia na kampusie itp.
	\item \textbf{Profile użytkowników:}
	      \\Każdy użytkownik powinien mieć profil z podstawowymi danymi (imię, nazwisko, nr indeksu, rocznik, itd.). Możliwość aktualizacji niektórych danych kontaktowych.

	\newpage

	\item \textbf{Responsywność i UX:}
	      \\Chcemy, żeby aplikacja była prosta i szybka w obsłudze.
	\item \textbf{Bezpieczeństwo danych: }
	      \\Chcemy, żeby dane były dobrze chronione, bo będą tu przechowywane prywatne informacje studentów. Może jakieś szyfrowanie?
	\item \textbf{Offline mode: }
	      \\Dobrze by było, gdyby część funkcji działała offline (np. podgląd planu zajęć lub ocen).
\end{itemize}
